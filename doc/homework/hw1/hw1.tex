\pagestyle{fancy}
\setlength{\headheight}{16pt}
\fancyhead{} % clear all header fields
\fancyhead[L]{\textbf{CEE 576 Homework 1}}
\fancyhead[C]{Songyuan Cui}
\fancyhead[R]{\textbf{Fall 2024}}
\fancyfoot{} % clear all footer fields
\fancyfoot[C]{\thepage}

\section{Newton-Raphson convergence}
Here we assume the algebraic error convergence recurrence formula 
\begin{equation}\label{eqn:hw1_p1_1}
    \| e_{n+1} \| \leq C \| e_n \|^k.
\end{equation}
where for simplicity we take equality. 
Given an initial error of $0.9$, the iterations taken to reach $\| e_n \| \leq 0.036$ are shown in \cref{tab:hw1_p1_1}.
For $C=1$, $k=1.1$, $37$ iterations are needed to bring the error below $0.036$, while for $C=0.5$, $k=1$ only $5$ are needed (\cref{tab:hw1_p1_1}). 

\begin{table}[!ht]
    \centering
    \begin{tabular}{|c|c|c|c|}
        \hline
        \multicolumn{2}{|c|}{$C=1$, $k=1.1$} & \multicolumn{2}{|c|}{$C=0.5$, $k=1$} \\
        \hline
        $n$ & $ \|e_n \| $ & $n$ & $\|e_n \|$ \\
        \hline
        0 & 0.9 & 0 & 0.9 \\
        \hline 
        1 & 0.8905673324 & 1 & 0.45 \\
        \hline 
        2 & 0.8803055429 & 2 & 0.225 \\
        \hline 
        3 & 0.8691540938 & 3 & 0.1125 \\
        \hline 
        4 & 0.8570505897 & 4 & 0.05625 \\
        \hline 
        5 & 0.8439313183 & \textcolor{red}{5} & \textcolor{red}{0.028125} \\
        \hline 
          & \ldots & 6 & 0.0140625 \\
        \hline 
        33 & 0.08655162295 & 7 & 0.00703125 \\
        \hline 
        34 & 0.06776457789 & 8 & 0.003515625 \\
        \hline 
        35 & 0.05177296187 & 9 & 0.0017578125 \\
        \hline 
        36 & 0.03850466172 & 10 & 0.00087890625 \\
        \hline 
        \textcolor{red}{37} & \textcolor{red}{0.02780127051} & 11 & 0.000439453125 \\
        \hline 
    \end{tabular}
    \caption{
        Error evolution based on \cref{eqn:hw1_p1_1}. Iterations where errors drop below $0.036$ are marked in red. 
        For conciseness, iterations 6--32 are omitted for $C = 1$, $k = 1.1$.
    }
    \label{tab:hw1_p1_1}
\end{table}

\section{Bar structure calculation}
The following parameters are given for this problem:
\begin{equation}
    L_a = \qty{10}{\centi\meter}, ~~ L_b = \qty{5}{\centi\meter}, ~~ A = \qty{1}{\centi\meter\squared}, ~~ E_e = \qty[per-mode=symbol]{1e+7}{\newton\per\centi\meter\squared}, ~~ E_p = \qty[per-mode=symbol]{1e+5}{\newton\per\centi\meter\squared}, ~~ \varepsilon_y = 2\times 10^{-3}
\end{equation}
where $L_a$ and $L_b$ are the left and right segment length, $A$ is the uniform cross-sectional area, $E_e$ and $E_p$ are the Young's modulus for the elastic and plastic regimes respectively, and $\epsilon_y$ is the yielding strain. 
This leads to a nonlinear consitutive relation which can be written as 
\begin{equation}
    \sigma(\varepsilon) = \begin{cases}
        E_e \varepsilon, & \textrm{if} ~~\varepsilon \leq \varepsilon_y \\
        E_e \varepsilon_y + E_p(\varepsilon - \varepsilon_y), & \textrm{if} ~~\varepsilon > \varepsilon_y
    \end{cases}
\end{equation}
An external force $F^{\textrm{ext}}$ is imposed at the intersection between segments $A$ and $B$, causing an unknown displacement $u$. 
The force balance at this point can be cast as a nonlinear problem of the form 
\begin{equation}\label{eqn:hw1_p2_fint}
    F^{\textrm{int}}(u) = \left[\sigma\left(\frac{u}{L_a}\right) + \sigma\left(\frac{u}{L_b}\right)\right] A = F^{\textrm{ext}}.
\end{equation}
Given the displacement and residual at the current iteration $u^{(i)}$ and $r^{(i)}$, the tangent operator $K$ (obtained at the start of the load for modified Newton's method), and the current external load $F^{\textrm{ext}}$, the iteration step is summarized as 
\begin{subequations}
\begin{equation}\label{eqn:hw1_p2_du}
    \Delta u^{(i)} = K^{-1} r^{(i)}
\end{equation}
\begin{equation}\label{eqn:hw1_p2_u_update}
    u^{(i+1)} = u^{(i)} + \Delta u^{(i)}
\end{equation}
\begin{equation}\label{eqn:hw1_p2_r_update}
    r^{(i+1)} = F^{\textrm{ext}} - F^{\textrm{int}}(u^{(i+1)})
\end{equation}
\end{subequations}
% The convergence of modified Newton-Raphson iterations is determined based on the relative residual functionally represented as 
% \begin{equation*}
%     \tilde{r}^{(i)}:= \frac{r^{(i)}}{r^{(0)}} \leq \epsilon ~~~~ \Leftrightarrow ~~~~ r^{(i)} \leq \epsilon r^{(0)}
% \end{equation*}
% where $r^{(0)}$ is the initial residual at the start of the load step. 

For the second load step ($F_2^{\textrm{ext}} = \qty{4e+4}{\newton}$), the modifed Newton-Raphson method uses the same tangent operator $K_2 = \qty[per-mode=symbol]{3e+6}{\newton\per\centi\meter}$. 
An absolute tolerance of $\epsilon = \qty{0.1}{\newton}$ is used (for residuals). 
The first iteration leads to displacement $u_2^{(1)} \approx \qty{1.3333e-2}{\centi\meter}$ and residual $r_2^{(1)} = \qty{6600}{\newton}$.
The second iteration leads to displacement $u_2^{(2)} \approx \qty{1.5533e-2}{\centi\meter}$ and residual $r_2^{(2)} = \qty{4356}{\newton}$.
Here, we explicitly use ``$=$'' for exact results and ``$\approx$'' for truncated results.
At this iteration, bar $A$ is elastic while bar $B$ is plastic. 


\emph{Iteration 3.} The incremental displacement can be evaluated using \cref{eqn:hw1_p2_du} as 
\begin{equation*}
    \Delta u^{(2)} = K_2^{-1} r_2^{(2)} = \frac{\qty{4356}{\newton}}{\qty[per-mode=symbol]{3e+6}{\newton\per\centi\meter}} = \qty{1.452e-3}{\centi\meter}.
\end{equation*}
The new displacement is then (\cref{eqn:hw1_p2_u_update})
\begin{equation*}
    u_2^{(3)} = u_2^{(2)} + \Delta u^{(2)} \approx \qty{1.5533e-2}{\centi\meter} +  \qty{1.452e-3}{\centi\meter} \approx \qty{1.6985e-2}{\centi\meter}.
\end{equation*}
The new residual, using \cref{eqn:hw1_p2_r_update,eqn:hw1_p2_fint}, is calculated as 
\begin{equation*}
    r_2^{(3)} = F_2^{\textrm{ext}} - F^{\textrm{int}}(u_2^{(3)}) \approx \qty{4e+4}{\newton} - \qty{3.7125e+4}{\newton} = \qty{2874.96}{\newton},
\end{equation*}
which is larger than the absolute tolerance $\epsilon = \qty{0.1}{\newton}$, suggesting more iterations are required. 

\emph{Iteration 4.} The incremental displacement can be evaluated using \cref{eqn:hw1_p2_du} as 
\begin{equation*}
    \Delta u^{(3)} = K_2^{-1} r_2^{(3)} = \frac{\qty{2874.96}{\newton}}{\qty[per-mode=symbol]{3e+6}{\newton\per\centi\meter}} \approx \qty{9.5832e-4}{\centi\meter}.
\end{equation*}
The new displacement is then (\cref{eqn:hw1_p2_u_update})
\begin{equation*}
    u_2^{(4)} = u_2^{(3)} + \Delta u^{(3)} \approx \qty{1.6985e-2}{\centi\meter} +  \qty{9.5832e-4}{\centi\meter} \approx \qty{1.7944e-2}{\centi\meter}.
\end{equation*}
The new residual, using \cref{eqn:hw1_p2_r_update,eqn:hw1_p2_fint}, is calculated as 
\begin{equation*}
    r_2^{(4)} = F_2^{\textrm{ext}} - F^{\textrm{int}}(u_2^{(4)}) \approx \qty{4e+4}{\newton} - \qty{3.8103e+4}{\newton} \approx \qty{1897.47}{\newton},
\end{equation*}
which is larger than the absolute tolerance $\epsilon = \qty{0.1}{\newton}$, suggesting more iterations are required. 

Key outputs of the the rest of the modifed Newton-Raphson iterations are given below. A total of 28 iterations are required for the residual magnitude $\| r_2^{(k)}\|$ to drop below $\epsilon = \qty{0.1}{\newton}$.
The final solution is $u_2 \approx \qty{0.01980}{\centi\meter}$, with bar $A$ still in the elastic regime and bar $B$ in the plastic regime. 

\begin{codenv}{Modified Newton-Raphson iterations at the second load step}
\indent iter=0, K=3e+06, u=6.666667e-03, Fint=2.000000e+04, res=2.000000e+04, du=6.666667e-03

iter=1, K=3e+06, u=1.333333e-02, Fint=3.340000e+04, res=6.600000e+03, du=2.200000e-03

iter=2, K=3e+06, u=1.553333e-02, Fint=3.564400e+04, res=4.356000e+03, du=1.452000e-03

iter=3, K=3e+06, u=1.698533e-02, Fint=3.712504e+04, res=2.874960e+03, du=9.583200e-04

iter=4, K=3e+06, u=1.794365e-02, Fint=3.810253e+04, res=1.897474e+03, du=6.324912e-04

iter=5, K=3e+06, u=1.857614e-02, Fint=3.874767e+04, res=1.252333e+03, du=4.174442e-04

iter=6, K=3e+06, u=1.899359e-02, Fint=3.917346e+04, res=8.265395e+02, du=2.755132e-04

iter=7, K=3e+06, u=1.926910e-02, Fint=3.945448e+04, res=5.455161e+02, du=1.818387e-04

iter=8, K=3e+06, u=1.945094e-02, Fint=3.963996e+04, res=3.600406e+02, du=1.200135e-04

iter=9, K=3e+06, u=1.957095e-02, Fint=3.976237e+04, res=2.376268e+02, du=7.920893e-05

iter=10, K=3e+06, u=1.965016e-02, Fint=3.984317e+04, res=1.568337e+02, du=5.227790e-05

iter=11, K=3e+06, u=1.970244e-02, Fint=3.989649e+04, res=1.035102e+02, du=3.450341e-05

iter=12, K=3e+06, u=1.973694e-02, Fint=3.993168e+04, res=6.831675e+01, du=2.277225e-05

iter=13, K=3e+06, u=1.975972e-02, Fint=3.995491e+04, res=4.508906e+01, du=1.502969e-05

iter=14, K=3e+06, u=1.977475e-02, Fint=3.997024e+04, res=2.975878e+01, du=9.919593e-06

iter=15, K=3e+06, u=1.978467e-02, Fint=3.998036e+04, res=1.964079e+01, du=6.546931e-06

iter=16, K=3e+06, u=1.979121e-02, Fint=3.998704e+04, res=1.296292e+01, du=4.320975e-06

iter=17, K=3e+06, u=1.979553e-02, Fint=3.999144e+04, res=8.555530e+00, du=2.851843e-06

iter=18, K=3e+06, u=1.979839e-02, Fint=3.999435e+04, res=5.646650e+00, du=1.882217e-06

iter=19, K=3e+06, u=1.980027e-02, Fint=3.999627e+04, res=3.726789e+00, du=1.242263e-06

iter=20, K=3e+06, u=1.980151e-02, Fint=3.999754e+04, res=2.459681e+00, du=8.198935e-07

iter=21, K=3e+06, u=1.980233e-02, Fint=3.999838e+04, res=1.623389e+00, du=5.411297e-07

iter=22, K=3e+06, u=1.980287e-02, Fint=3.999893e+04, res=1.071437e+00, du=3.571456e-07

iter=23, K=3e+06, u=1.980323e-02, Fint=3.999929e+04, res=7.071483e-01, du=2.357161e-07

iter=24, K=3e+06, u=1.980346e-02, Fint=3.999953e+04, res=4.667179e-01, du=1.555726e-07

iter=25, K=3e+06, u=1.980362e-02, Fint=3.999969e+04, res=3.080338e-01, du=1.026779e-07

iter=26, K=3e+06, u=1.980372e-02, Fint=3.999980e+04, res=2.033023e-01, du=6.776744e-08

iter=27, K=3e+06, u=1.980379e-02, Fint=3.999987e+04, res=1.341795e-01, du=4.472651e-08

iter=28, K=3e+06, u=1.980383e-02, Fint=3.999987e+04, res=8.855849e-02, du=4.472651e-08

\end{codenv}

\section{Bar structure computation}
Following the previous problem, the nonlinear equation \cref{eqn:hw1_p2_fint} can be solved via either the Newton-Raphson (N-R) method or the Modified Newton-Raphson (M-N-R) method. 
The general algorithm for these approaches are listed in \cref{alg:hw1_nr,alg:hw1_mnr}, respectively. 
A representitive flow chart is shown in \cref{fig:hw1_flow_chart}.
We note that the external loads are applied in incremental load steps denoted in \emph{subscripts}, while the paranthesized \emph{superscripts} correspond to Newton iterations.
For example, $u_n^{(k)}$ represent the $k$-th iteration during the $n$-th load step. 

\begin{algorithm}[!ht]
\caption{Nonlinear solution procedure with Newton-Raphson iteration}
\begin{algorithmic}\label{alg:hw1_nr}
    \setlength{\lineskip}{4pt}
    \Require{Nonlinear function $F^{\textrm{int}}(u)$, incremental load steps $F^{\textrm{ext}}_n$, Initial guess $u_0$, Tolerance $\epsilon$}
    \Ensure{Approximated solutions such that $F^{\textrm{int}}(u_n) \approx F^{\textrm{ext}}_n$}

    \For{$n = 1, 2, \ldots, N$}
    \Comment{New load step}

    \State{$u_n^{(0)} \gets u_{n-1}$}
    \Comment{Inherit solution from previous step}

    \State{$r_n^{(0)} \gets F^{\textrm{ext}}_n - F^{\textrm{int}} \left[u_n^{(0)}\right]$}
    \Comment{Compute new residual}

    \State{$k \gets 0$}

    \While{$\| r_n^{(k)}\| > \epsilon$}
        \State{\textcolor{Maroon}{$K_n^{(k)} \gets \frac{\partial}{\partial u} F^{\textrm{int}}\left[u_n^{(k)}\right]$}}
        \Comment{Compute consistent tangent}

        \State{$\Delta u_n^{(k)} = {\left[K_n^{(k)}\right]}^{-1}r_n^{(k)}$}
        \Comment{Compute incremental displacement}

        \State{$u_n^{(k+1)} \gets u_n^{(k)} + \Delta u_n^{(k)}$}
        \Comment{Update displacement}

        \State{$r_n^{(k+1)} \gets F^{\textrm{ext}}_n - F^{\textrm{int}} \left[u_n^{(k+1)}\right]$}
        \Comment{Update residual}

        \State{$k \gets k + 1$}
    \EndWhile{}

    \State{$u_n \gets u_n^{(k)}$}
    \Comment{Solution of the $n$-th load step}

    \EndFor{}
\end{algorithmic}
\end{algorithm}


\begin{algorithm}[!ht]
\caption{Nonlinear solution procedure with Modified Newton-Raphson iteration}
\begin{algorithmic}\label{alg:hw1_mnr}
    \setlength{\lineskip}{4pt}
    \Require{Nonlinear function $F^{\textrm{int}}(u)$, incremental load steps $F^{\textrm{ext}}_n$, Initial guess $u_0$, Tolerance $\epsilon$}
    \Ensure{Approximated solutions such that $F^{\textrm{int}}(u_n) \approx F^{\textrm{ext}}_n$}

    \For{$n = 1, 2, \ldots, N$}
    \Comment{New load step}

    \State{$u_n^{(0)} \gets u_{n-1}$}
    \Comment{Inherit solution from previous step}

    \State{$r_n^{(0)} \gets F^{\textrm{ext}}_n - F^{\textrm{int}} \left[u_n^{(0)}\right]$}
    \Comment{Compute new residual}

    \State{$k \gets 0$}

    \State{\textcolor{Maroon}{$K_n \gets \frac{\partial}{\partial u} F^{\textrm{int}}\left[u_n^{(0)}\right]$}}
    \Comment{Compute consistent tangent only at $k = 0$}

    \While{$\| r_n^{(k)}\| > \epsilon$}
        
        \State{$\Delta u_n^{(k)} = K_n^{-1}r_n^{(k)}$}
        \Comment{Compute incremental displacement}

        \State{$u_n^{(k+1)} \gets u_n^{(k)} + \Delta u_n^{(k)}$}
        \Comment{Update displacement}

        \State{$r_n^{(k+1)} \gets F^{\textrm{ext}}_n - F^{\textrm{int}} \left[u_n^{(k+1)}\right]$}
        \Comment{Update residual}

        \State{$k \gets k + 1$}
    \EndWhile{}

    \State{$u_n \gets u_n^{(k)}$}
    \Comment{Solution of the $n$-th load step}

    \EndFor{}
\end{algorithmic}
\end{algorithm}

\begin{figure}[!ht]
    \centering
    \includegraphics[width=\linewidth]{homework/hw1/hw1_flow_chart.pdf}
    \caption{Flow diagram of Newton-Raphson and modifed Newton-Raphson iterations. }
    \label{fig:hw1_flow_chart}
\end{figure}

The two algorithms are largely identical, with the only difference in where the consistent tangent is computed (marked in \textcolor{Maroon}{dark red}).
The N-R method compute the consistent tangent at every iteration, while M-N-R do this only at the start of a load step and the same tangent is used throughout the step. 

Numerical implementation of both N-R and M-N-R is accomplished in \matlab. 
The force-displacement curve, with two successive load steps $F^{\textrm{ext}}_1 = \qty{2e+4}{\newton}$ and $F^{\textrm{ext}}_2 = \qty{4e+4}{\newton}$, is shown in 

\begin{figure}[!ht]
    \centering
    \includegraphics[width=0.6\linewidth]{homework/hw1/hw1_p3.pdf}
    \caption{Exact force-displacement curve and computation results using N-R (red dots) and M-N-R (blue dots) iterations.}
    \label{fig:hw1_p3_fd}
\end{figure}
